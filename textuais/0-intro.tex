\section{Introdução}

    Empresas, instituições acadêmicas, cidadãos e profissionais de diversas áreas frequentemente
se deparam com a necessidade de realizar cálculos extensos e complexos, que são cruciais para tomada
de decisões, análises financeiras, pesquisas científicas, otimização de processos e diversas outras
aplicações. No entanto, enfrentar esse tipo de tarefa apresenta vários obstáculos que podem
comprometer a eficiência e a precisão dos resultados.
    Erros humanos são uma preocupação constante ao lidar com cálculos, seja manualmente ou ao
programá-los. A maior propensão a erros compromete a acuracidade dos resultados e as revisões e
correções também demandam tempo adicional. A falta de ferramentas adequadas que facilitem a
automação e execução eficiente desses cálculos é outro entrave significativo. Muitas vezes, as
ferramentas disponíveis não são adaptadas às necessidades específicas do usuário.
    Além disso, a complexidade algorítmica dos cálculos demanda uma agilidade maior do
usuário e pode ser difícil de otimizar. Implementações ineficientes dos algoritmos podem aumentar
ainda mais o tempo necessário para resolver os cálculos.
    Em um mundo onde as interações sociais e financeiras estão cada vez mais digitalizadas, a
necessidade de ferramentas que facilitem a gestão de despesas compartilhadas torna-se evidente.
Eventos sociais, viagens, moradias compartilhadas e muitas outras situações cotidianas
envolvem a divisão de contas entre várias pessoas. A ausência de uma ferramenta eficaz para
gerenciar essas divisões pode levar a conflitos, falta de transparência e dificuldades
financeiras.
A aplicação proposta busca solucionar um desafio comum em situações de convívio social, onde múltiplas pessoas compartilham despesas, como em compras coletivas ou eventos em grupo. A divisão justa dos custos é fundamental para evitar conflitos e garantir que todos contribuam de forma proporcional ao que consumiram. Utilizando tecnologia de ponta, a aplicação permite automatizar o processo de cálculo das contribuições, assegurando transparência e equidade na divisão dos valores. Além disso, a solução proporciona uma interface intuitiva, facilitando o uso por qualquer pessoa, independentemente do seu nível de familiaridade com ferramentas digitais. A capacidade de gerenciar pagamentos, registrar contribuições e fornecer relatórios detalhados torna a aplicação uma ferramenta essencial para grupos que buscam uma forma organizada e justa de dividir suas despesas.